\documentclass{article}
\usepackage{graphicx}
\graphicspath{ {c:/users/Nils Becker/downloads/} }
\usepackage{amsmath}

\title{Report Assignment IV - Dynamics}
\date{Fundamentals of Robotics - FS 2021\\-\\December 9$^{th}$ 2021}
\author{Nils Becker}

\begin{document}
  \maketitle
  \newpage
  \section{Euler-Lagrange}
	\begin{equation}
	\tau = M(q)\ddot{q}+c(q,\dot{q})\dot{q}+g(q)
	\end{equation}
	\subsection{Inertia Matrix M(q)}
	\begin{equation}
	M(q) = \sum^6_{i=1}m_iJ_{v_i}^TJ_{v_i}+(J_{w_i}^TR_i)I_i(R_i^TJ_{w_i})
	\end{equation}
	Note that $M(q)$ will be of shape 6x6, because the robot has 6 DoF.
  \subsubsection{Finding CoM's}

  We find:
  \begin{equation}
  cm_1=R_z(q_1)lc_1
  \end{equation}
  \begin{equation}
  cm_2=R_z(q_1)R_y(q_2)T(pos_2)lc_2  
  \end{equation}
  \begin{equation}
  cm_3=R_z(q_1)R_y(q_2)R_y(q_3)T(pos_3)lc_3
  \end{equation}
  \begin{equation}
  cm_4=R_z(q_1)R_y(q_2)R_y(q_3)R_z(q_4)T(pos_4)lc_4
  \end{equation}
  \begin{equation}
  cm_5=R_z(q_1)R_y(q_2)R_y(q_3)R_z(q_4)Rx(q_5)T(pos_6)lc_5
  \end{equation}\begin{equation}
  cm_6=R_z(q_1)R_y(q_2)R_y(q_3)R_z(q_4)R_x(q_5)R_z(q_6)T(pos_6)lc_6
  \end{equation}
  $pos_i$ represents the position of joint $i$. Note that $T(pos_1)=(0,0,0)^T$, thus it can be ignored in (3). \\
  $lc_i$ represents the vector from joint $i-1$ to the CoM of link $i$\\
  $l_i$ represents the vector from joint $i-1$ to joint $i$

\subsubsection{Finding the Jacobians}
The Jacobian for linear velocity is given by:
\begin{equation}
	J_{v_i} = \begin{pmatrix}
	\frac{\partial x_i}{\partial q_1}&\frac{\partial x_i}{\partial q_2}&\frac{\partial x_i}{\partial q_3}&\frac{\partial x_i}{\partial q_4}&\frac{\partial x_i}{\partial q_5}&\frac{\partial x_i}{\partial q_6}\\
	\frac{\partial y_i}{\partial q_1}&\frac{\partial y_i}{\partial q_2}&\frac{\partial y_i}{\partial q_3}&\frac{\partial y_i}{\partial q_4}&\frac{\partial y_i}{\partial q_4}&\frac{\partial y_i}{\partial q_6}\\
	\frac{\partial z_i}{\partial q_1}&\frac{\partial z_i}{\partial q_2}&\frac{\partial z_i}{\partial q_3}&\frac{\partial z_i}{\partial q_4}&\frac{\partial z_i}{\partial q_5}&\frac{\partial z_i}{\partial q_6}
	\end{pmatrix}
\end{equation}
Note that:
\begin{equation}
cm_i = \begin{pmatrix}
	x_i\\y_i\\z_i
	\end{pmatrix}
\end{equation}
The Jacobian for angular velocities is given by:
\begin{equation}
J_{w_i} = \begin{pmatrix}
	u_{0}&u_{1}&...&u_{n-1}
	\end{pmatrix}
\end{equation}
Where $u_i$ represents the axis of rotation of the joint. $J_{w_i}$ will be of shape 6x3 for all $i$.
\subsubsection{Matrix I}
The Matrix $I$ for each link will be of form:
\begin{equation}
I_i = \begin{pmatrix}
	I_{xx_i}&&\\
	&I_{yy_i}&\\
	&&I_{zz_i}&	\end{pmatrix}
\end{equation}
Seeing evey link as a cuboid with length $a$, height $b$ and depth $c$.
\subsection{Centrifugal- and Corelolis Forces }
The $c(q,\dot{q})$ will also be of size 6x6. Its components will be calculated like this:
\begin{equation}
\begin{array}{ll}
		c_{ij} = \sum_{k=1}^6c_{ijk}*\dot{q}  &, i,j \in 1,2,...,6
\end{array}
\end{equation}
Where:
\begin{equation}
c_{ijk} = \frac{1}{2}(\frac{\partial M_{ij}}{\partial q_k}+\frac{\partial M_{ik}}{\partial q_j}-\frac{\partial M_{jk}}{\partial q_i})
\end{equation}
\subsection{Gravity}
We get the last vector of the equation as:
\begin{equation}
g(q) = \begin{pmatrix}
	g_1\\
	g_2\\
	...\\
	g_6
	\end{pmatrix}
\end{equation}
Where we calculate:
\begin{equation}
g_i = -\sum_{k=1}^6J_{v_k}^im_kg_0
\end{equation}
Where $J_{v_k}^i$ represents the $i^th$  column of the $k^th$ linear velocity matrix.\\
And:
\begin{equation}
g_0 = \begin{pmatrix}
		0\\
		0\\
		-9.81
	\end{pmatrix}
\end{equation}
representing the gravity in opposite direction of the z-axis.
\subsection{Annotation}
As the terms get humongous pretty fast, they are not shown in the report. To see the complete model use $print(s\_euler\_lagrange())$ in the code. To see $M(q)$, $print(c)$ and $g(q)$ use $print(mq[1])$ , $cprint(c)$ and $print(g)$ inside $s\_euler\_lagrange()$.\\ All the commands will be commented out in the submitted code.
\newpage
\section{Code Implementation}
\subsection{Use of SymPy}
The usage of this Python Toolbox allows for symbolical computations of all the matrices and partial deriavtives. The possibility to give variables predefined names proved to be highly advantageous in keeping track of the calculations. In general the used approach can be used for any configuration of $q_1-q_6$, $l_1-l_6$ and $lc_1-lc_6$ as these symbolical values can be swapped out with numerical ones and the code will return the assiciated $\tau$.
\subsection{M(q)}
\subsubsection{Calculating the CoM's}
As shown in section 1.1.1 the CoM are calculated through a sequence of matrix multipicaltions and returned in a list. The function, which calculated these CoM should not be called with numerical values, as it's existance is purely for the sake of deriving them later in the Jacobians.
\subsubsection{Calculating linear Jacobians}
Using the previously calculated CoM's, this function derives the points, by iterating through the CoM's. For each point, it iterates through $q_1-q_6$ and for each $q_i$ through $x,y,z$. The results will be put into matrix representing $J_{v_i}$, maintaining the structure shown in (9). All the $J_{v_i}$ will be put into a list, which will be the output of the function.
\subsubsection{Calculating angular Jacobians}
Firstly the forward kinematics of the manipulator are calculated symbolically, to get a transformation matrix from the base frame to all local frames. Using these transformations, all the $u_i$ can be calculated like in diffential kinematics.
\subsubsection{Inertia I Matrix}
Using SymPy, we get $I_i$. for each link.
\subsubsection{Putting it all together}
Using the six caluclated $J_v$, the six $J_w$ and the six $I$ we can use symbolic $3x3$ rotations matrices and -masses $m_1-m_6$ to complete the equation in (2). as the $M(a)$ has to be symetrically, the programm will check this after calculation, verifying the correctness of the implentation.
\newpage
\subsection{C(q,dq)}
\subsubsection{Christoffel Symbol}
Here the program will apply (14) with $M,i,j,k$ being parameters of the function. Note that the calculation of $c_{ijk}$ might take some time, as the terms to be derived are complex.
\subsubsection{Generating C Matrix}
Here an 6x6 matrix, filled with zeros will be the starting point. Then all the $c_{ij}$ are calculated according to (13) and put into the correct spot into the resulting matrix.
\subsection{G(q)}
In the corresponding function, the list of all $J_{v_i}$ is calculated. Iterating over $m_1-m_6$ the 6x1 Matrix is calculated following (16).
\end{document}